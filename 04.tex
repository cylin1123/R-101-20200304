% Options for packages loaded elsewhere
\PassOptionsToPackage{unicode}{hyperref}
\PassOptionsToPackage{hyphens}{url}
%
\documentclass[
]{article}
\usepackage{lmodern}
\usepackage{amssymb,amsmath}
\usepackage{ifxetex,ifluatex}
\ifnum 0\ifxetex 1\fi\ifluatex 1\fi=0 % if pdftex
  \usepackage[T1]{fontenc}
  \usepackage[utf8]{inputenc}
  \usepackage{textcomp} % provide euro and other symbols
\else % if luatex or xetex
  \usepackage{unicode-math}
  \defaultfontfeatures{Scale=MatchLowercase}
  \defaultfontfeatures[\rmfamily]{Ligatures=TeX,Scale=1}
\fi
% Use upquote if available, for straight quotes in verbatim environments
\IfFileExists{upquote.sty}{\usepackage{upquote}}{}
\IfFileExists{microtype.sty}{% use microtype if available
  \usepackage[]{microtype}
  \UseMicrotypeSet[protrusion]{basicmath} % disable protrusion for tt fonts
}{}
\makeatletter
\@ifundefined{KOMAClassName}{% if non-KOMA class
  \IfFileExists{parskip.sty}{%
    \usepackage{parskip}
  }{% else
    \setlength{\parindent}{0pt}
    \setlength{\parskip}{6pt plus 2pt minus 1pt}}
}{% if KOMA class
  \KOMAoptions{parskip=half}}
\makeatother
\usepackage{xcolor}
\IfFileExists{xurl.sty}{\usepackage{xurl}}{} % add URL line breaks if available
\IfFileExists{bookmark.sty}{\usepackage{bookmark}}{\usepackage{hyperref}}
\hypersetup{
  pdftitle={R物件資料結構},
  pdfauthor={Nick Lin},
  hidelinks,
  pdfcreator={LaTeX via pandoc}}
\urlstyle{same} % disable monospaced font for URLs
\usepackage[margin=1in]{geometry}
\usepackage{color}
\usepackage{fancyvrb}
\newcommand{\VerbBar}{|}
\newcommand{\VERB}{\Verb[commandchars=\\\{\}]}
\DefineVerbatimEnvironment{Highlighting}{Verbatim}{commandchars=\\\{\}}
% Add ',fontsize=\small' for more characters per line
\usepackage{framed}
\definecolor{shadecolor}{RGB}{248,248,248}
\newenvironment{Shaded}{\begin{snugshade}}{\end{snugshade}}
\newcommand{\AlertTok}[1]{\textcolor[rgb]{0.94,0.16,0.16}{#1}}
\newcommand{\AnnotationTok}[1]{\textcolor[rgb]{0.56,0.35,0.01}{\textbf{\textit{#1}}}}
\newcommand{\AttributeTok}[1]{\textcolor[rgb]{0.77,0.63,0.00}{#1}}
\newcommand{\BaseNTok}[1]{\textcolor[rgb]{0.00,0.00,0.81}{#1}}
\newcommand{\BuiltInTok}[1]{#1}
\newcommand{\CharTok}[1]{\textcolor[rgb]{0.31,0.60,0.02}{#1}}
\newcommand{\CommentTok}[1]{\textcolor[rgb]{0.56,0.35,0.01}{\textit{#1}}}
\newcommand{\CommentVarTok}[1]{\textcolor[rgb]{0.56,0.35,0.01}{\textbf{\textit{#1}}}}
\newcommand{\ConstantTok}[1]{\textcolor[rgb]{0.00,0.00,0.00}{#1}}
\newcommand{\ControlFlowTok}[1]{\textcolor[rgb]{0.13,0.29,0.53}{\textbf{#1}}}
\newcommand{\DataTypeTok}[1]{\textcolor[rgb]{0.13,0.29,0.53}{#1}}
\newcommand{\DecValTok}[1]{\textcolor[rgb]{0.00,0.00,0.81}{#1}}
\newcommand{\DocumentationTok}[1]{\textcolor[rgb]{0.56,0.35,0.01}{\textbf{\textit{#1}}}}
\newcommand{\ErrorTok}[1]{\textcolor[rgb]{0.64,0.00,0.00}{\textbf{#1}}}
\newcommand{\ExtensionTok}[1]{#1}
\newcommand{\FloatTok}[1]{\textcolor[rgb]{0.00,0.00,0.81}{#1}}
\newcommand{\FunctionTok}[1]{\textcolor[rgb]{0.00,0.00,0.00}{#1}}
\newcommand{\ImportTok}[1]{#1}
\newcommand{\InformationTok}[1]{\textcolor[rgb]{0.56,0.35,0.01}{\textbf{\textit{#1}}}}
\newcommand{\KeywordTok}[1]{\textcolor[rgb]{0.13,0.29,0.53}{\textbf{#1}}}
\newcommand{\NormalTok}[1]{#1}
\newcommand{\OperatorTok}[1]{\textcolor[rgb]{0.81,0.36,0.00}{\textbf{#1}}}
\newcommand{\OtherTok}[1]{\textcolor[rgb]{0.56,0.35,0.01}{#1}}
\newcommand{\PreprocessorTok}[1]{\textcolor[rgb]{0.56,0.35,0.01}{\textit{#1}}}
\newcommand{\RegionMarkerTok}[1]{#1}
\newcommand{\SpecialCharTok}[1]{\textcolor[rgb]{0.00,0.00,0.00}{#1}}
\newcommand{\SpecialStringTok}[1]{\textcolor[rgb]{0.31,0.60,0.02}{#1}}
\newcommand{\StringTok}[1]{\textcolor[rgb]{0.31,0.60,0.02}{#1}}
\newcommand{\VariableTok}[1]{\textcolor[rgb]{0.00,0.00,0.00}{#1}}
\newcommand{\VerbatimStringTok}[1]{\textcolor[rgb]{0.31,0.60,0.02}{#1}}
\newcommand{\WarningTok}[1]{\textcolor[rgb]{0.56,0.35,0.01}{\textbf{\textit{#1}}}}
\usepackage{graphicx,grffile}
\makeatletter
\def\maxwidth{\ifdim\Gin@nat@width>\linewidth\linewidth\else\Gin@nat@width\fi}
\def\maxheight{\ifdim\Gin@nat@height>\textheight\textheight\else\Gin@nat@height\fi}
\makeatother
% Scale images if necessary, so that they will not overflow the page
% margins by default, and it is still possible to overwrite the defaults
% using explicit options in \includegraphics[width, height, ...]{}
\setkeys{Gin}{width=\maxwidth,height=\maxheight,keepaspectratio}
% Set default figure placement to htbp
\makeatletter
\def\fps@figure{htbp}
\makeatother
\setlength{\emergencystretch}{3em} % prevent overfull lines
\providecommand{\tightlist}{%
  \setlength{\itemsep}{0pt}\setlength{\parskip}{0pt}}
\setcounter{secnumdepth}{-\maxdimen} % remove section numbering

\title{R物件資料結構}
\author{Nick Lin}
\date{2020/1/31}

\begin{document}
\maketitle

\hypertarget{ux5411ux91cfvector}{%
\subsubsection{01.向量Vector}\label{ux5411ux91cfvector}}

\begin{itemize}
\tightlist
\item
  必須由相同的資料型態元素所組成
\item
  建立向量 : 使用c函數 / 使用assign函數 / 使用「:」
\item
  向量取值 : 變數名稱{[}index{]},index起始值為1
\end{itemize}

\begin{Shaded}
\begin{Highlighting}[]
\CommentTok{#使用c函數建立向量}
\NormalTok{v1 <-}\StringTok{ }\KeywordTok{c}\NormalTok{(}\DecValTok{5}\NormalTok{,}\FloatTok{7.5}\NormalTok{,}\DecValTok{10}\NormalTok{,}\FloatTok{12.5}\NormalTok{,}\DecValTok{15}\NormalTok{,}\DecValTok{25}\NormalTok{,}\DecValTok{20}\NormalTok{,}\FloatTok{17.5}\NormalTok{)}
\NormalTok{v1}
\end{Highlighting}
\end{Shaded}

\begin{verbatim}
## [1]  5.0  7.5 10.0 12.5 15.0 25.0 20.0 17.5
\end{verbatim}

\begin{Shaded}
\begin{Highlighting}[]
\KeywordTok{length}\NormalTok{(v1)}
\end{Highlighting}
\end{Shaded}

\begin{verbatim}
## [1] 8
\end{verbatim}

\begin{Shaded}
\begin{Highlighting}[]
\KeywordTok{mode}\NormalTok{(v1)}
\end{Highlighting}
\end{Shaded}

\begin{verbatim}
## [1] "numeric"
\end{verbatim}

\begin{Shaded}
\begin{Highlighting}[]
\CommentTok{#使用assign函數建立向量}
\KeywordTok{assign}\NormalTok{(}\StringTok{"v2"}\NormalTok{,}\KeywordTok{c}\NormalTok{(}\DecValTok{10}\NormalTok{,}\DecValTok{20}\NormalTok{,}\FloatTok{15.5}\NormalTok{,}\FloatTok{17.2}\NormalTok{,}\DecValTok{3}\NormalTok{,}\DecValTok{6}\NormalTok{,}\DecValTok{1}\NormalTok{))}
\NormalTok{v2}
\end{Highlighting}
\end{Shaded}

\begin{verbatim}
## [1] 10.0 20.0 15.5 17.2  3.0  6.0  1.0
\end{verbatim}

\begin{Shaded}
\begin{Highlighting}[]
\KeywordTok{length}\NormalTok{(v2)}
\end{Highlighting}
\end{Shaded}

\begin{verbatim}
## [1] 7
\end{verbatim}

\begin{Shaded}
\begin{Highlighting}[]
\KeywordTok{mode}\NormalTok{(v2)}
\end{Highlighting}
\end{Shaded}

\begin{verbatim}
## [1] "numeric"
\end{verbatim}

\begin{Shaded}
\begin{Highlighting}[]
\CommentTok{#使用:建立向量}
\NormalTok{v3 =}\StringTok{ }\DecValTok{1}\OperatorTok{:}\DecValTok{10}
\NormalTok{v3}
\end{Highlighting}
\end{Shaded}

\begin{verbatim}
##  [1]  1  2  3  4  5  6  7  8  9 10
\end{verbatim}

\begin{Shaded}
\begin{Highlighting}[]
\KeywordTok{length}\NormalTok{(v3)}
\end{Highlighting}
\end{Shaded}

\begin{verbatim}
## [1] 10
\end{verbatim}

\begin{Shaded}
\begin{Highlighting}[]
\KeywordTok{mode}\NormalTok{(v3)}
\end{Highlighting}
\end{Shaded}

\begin{verbatim}
## [1] "numeric"
\end{verbatim}

\begin{Shaded}
\begin{Highlighting}[]
\KeywordTok{seq}\NormalTok{(}\DecValTok{2}\NormalTok{, }\DecValTok{5}\NormalTok{)  }
\end{Highlighting}
\end{Shaded}

\begin{verbatim}
## [1] 2 3 4 5
\end{verbatim}

\begin{Shaded}
\begin{Highlighting}[]
\KeywordTok{seq}\NormalTok{(}\DecValTok{2}\NormalTok{, }\DecValTok{5}\NormalTok{, }\DataTypeTok{by =} \FloatTok{0.5}\NormalTok{)}
\end{Highlighting}
\end{Shaded}

\begin{verbatim}
## [1] 2.0 2.5 3.0 3.5 4.0 4.5 5.0
\end{verbatim}

\begin{Shaded}
\begin{Highlighting}[]
\KeywordTok{seq}\NormalTok{(}\DecValTok{2}\NormalTok{, }\DecValTok{5}\NormalTok{, }\DataTypeTok{length =} \DecValTok{5}\NormalTok{)}
\end{Highlighting}
\end{Shaded}

\begin{verbatim}
## [1] 2.00 2.75 3.50 4.25 5.00
\end{verbatim}

\begin{Shaded}
\begin{Highlighting}[]
\KeywordTok{c}\NormalTok{(}\DecValTok{1}\OperatorTok{:}\DecValTok{4}\NormalTok{, }\DecValTok{8}\NormalTok{, }\DecValTok{9}\NormalTok{, }\KeywordTok{c}\NormalTok{(}\DecValTok{12}\NormalTok{, }\DecValTok{23}\NormalTok{))}
\end{Highlighting}
\end{Shaded}

\begin{verbatim}
## [1]  1  2  3  4  8  9 12 23
\end{verbatim}

vector 函數可以用來建立特定類型與長度的向量

說明: 使用vector 所建立的向量,其內部的值都是 0、FALSE 或 NULL
這類的空值

\begin{Shaded}
\begin{Highlighting}[]
\KeywordTok{vector}\NormalTok{(}\StringTok{"numeric"}\NormalTok{, }\DecValTok{3}\NormalTok{)}
\end{Highlighting}
\end{Shaded}

\begin{verbatim}
## [1] 0 0 0
\end{verbatim}

\begin{Shaded}
\begin{Highlighting}[]
\KeywordTok{vector}\NormalTok{(}\StringTok{"logical"}\NormalTok{, }\DecValTok{3}\NormalTok{)}
\end{Highlighting}
\end{Shaded}

\begin{verbatim}
## [1] FALSE FALSE FALSE
\end{verbatim}

\begin{Shaded}
\begin{Highlighting}[]
\KeywordTok{vector}\NormalTok{(}\StringTok{"character"}\NormalTok{, }\DecValTok{3}\NormalTok{)}
\end{Highlighting}
\end{Shaded}

\begin{verbatim}
## [1] "" "" ""
\end{verbatim}

\begin{Shaded}
\begin{Highlighting}[]
\KeywordTok{vector}\NormalTok{(}\StringTok{"list"}\NormalTok{, }\DecValTok{3}\NormalTok{)}
\end{Highlighting}
\end{Shaded}

\begin{verbatim}
## [[1]]
## NULL
## 
## [[2]]
## NULL
## 
## [[3]]
## NULL
\end{verbatim}

length 也可以讓使用者直接改變向量的長度屬性

用途 :
在處理大量運算,需要預先配置記憶體時,有時候就會使用增加向量長度的方式來處理

\begin{Shaded}
\begin{Highlighting}[]
\NormalTok{x <-}\StringTok{ }\DecValTok{1}\OperatorTok{:}\DecValTok{10}
\KeywordTok{length}\NormalTok{(x) <-}\StringTok{ }\DecValTok{3}
\NormalTok{x}
\end{Highlighting}
\end{Shaded}

\begin{verbatim}
## [1] 1 2 3
\end{verbatim}

\begin{Shaded}
\begin{Highlighting}[]
\KeywordTok{length}\NormalTok{(x) <-}\StringTok{ }\DecValTok{6}
\NormalTok{x}
\end{Highlighting}
\end{Shaded}

\begin{verbatim}
## [1]  1  2  3 NA NA NA
\end{verbatim}

\hypertarget{ux5411ux91cfux5143ux7d20ux540dux7a31}{%
\subparagraph{向量元素名稱}\label{ux5411ux91cfux5143ux7d20ux540dux7a31}}

\begin{itemize}
\tightlist
\item
  每一個元素都可以有一個自己的名稱
\item
  使用 name = value 的方式指定元素的名稱
\end{itemize}

\begin{Shaded}
\begin{Highlighting}[]
\KeywordTok{c}\NormalTok{(}\DataTypeTok{foo =} \DecValTok{2}\NormalTok{, }\DataTypeTok{bar =} \DecValTok{4}\NormalTok{)}
\end{Highlighting}
\end{Shaded}

\begin{verbatim}
## foo bar 
##   2   4
\end{verbatim}

\begin{Shaded}
\begin{Highlighting}[]
\KeywordTok{c}\NormalTok{(}\DataTypeTok{foo =} \DecValTok{2}\NormalTok{, }\DataTypeTok{bar =} \DecValTok{4}\NormalTok{, }\StringTok{"hello world"}\NormalTok{ =}\StringTok{ }\DecValTok{6}\NormalTok{, }\DecValTok{8}\NormalTok{)}
\end{Highlighting}
\end{Shaded}

\begin{verbatim}
##         foo         bar hello world             
##           2           4           6           8
\end{verbatim}

\begin{Shaded}
\begin{Highlighting}[]
\NormalTok{x <-}\StringTok{ }\DecValTok{1}\OperatorTok{:}\DecValTok{4}
\KeywordTok{names}\NormalTok{(x) <-}\StringTok{ }\KeywordTok{c}\NormalTok{(}\StringTok{"foo"}\NormalTok{, }\StringTok{"bar"}\NormalTok{, }\StringTok{"hello world"}\NormalTok{, }\StringTok{""}\NormalTok{)}
\NormalTok{x}
\end{Highlighting}
\end{Shaded}

\begin{verbatim}
##         foo         bar hello world             
##           1           2           3           4
\end{verbatim}

\begin{Shaded}
\begin{Highlighting}[]
\KeywordTok{names}\NormalTok{(x)}
\end{Highlighting}
\end{Shaded}

\begin{verbatim}
## [1] "foo"         "bar"         "hello world" ""
\end{verbatim}

\begin{Shaded}
\begin{Highlighting}[]
\KeywordTok{names}\NormalTok{(}\DecValTok{1}\OperatorTok{:}\DecValTok{10}\NormalTok{)}
\end{Highlighting}
\end{Shaded}

\begin{verbatim}
## NULL
\end{verbatim}

\hypertarget{ux7d22ux5f15ux5411ux91cf}{%
\paragraph{索引向量}\label{ux7d22ux5f15ux5411ux91cf}}

\begin{itemize}
\tightlist
\item
  使用{[}{]}
\item
  邏輯向量、正整數向量、負整數向量與字元向量
\end{itemize}

\begin{Shaded}
\begin{Highlighting}[]
\NormalTok{v3 <-}\StringTok{ }\DecValTok{1}\OperatorTok{:}\DecValTok{10}
\NormalTok{v3}
\end{Highlighting}
\end{Shaded}

\begin{verbatim}
##  [1]  1  2  3  4  5  6  7  8  9 10
\end{verbatim}

\begin{Shaded}
\begin{Highlighting}[]
\NormalTok{v3[}\DecValTok{3}\NormalTok{]}
\end{Highlighting}
\end{Shaded}

\begin{verbatim}
## [1] 3
\end{verbatim}

\begin{Shaded}
\begin{Highlighting}[]
\NormalTok{v3[}\DecValTok{1}\OperatorTok{:}\DecValTok{5}\NormalTok{]}
\end{Highlighting}
\end{Shaded}

\begin{verbatim}
## [1] 1 2 3 4 5
\end{verbatim}

\hypertarget{ux908fux8f2fux5411ux91cf}{%
\subparagraph{邏輯向量}\label{ux908fux8f2fux5411ux91cf}}

說明 : 索引向量必須和被挑選元素的向量長度一致。向量中對應索引向量為 TRUE
的元素將會被選出,而那些對應 FALSE 的元素則被忽略

\begin{Shaded}
\begin{Highlighting}[]
\NormalTok{x <-}\StringTok{ }\KeywordTok{c}\NormalTok{(}\DecValTok{0}\NormalTok{,}\DecValTok{1}\NormalTok{, }\OtherTok{NA}\NormalTok{, }\DecValTok{3}\NormalTok{,}\DecValTok{4}\NormalTok{)}
\NormalTok{y <-}\StringTok{ }\NormalTok{x[}\OperatorTok{!}\KeywordTok{is.na}\NormalTok{(x)]}
\NormalTok{y}
\end{Highlighting}
\end{Shaded}

\begin{verbatim}
## [1] 0 1 3 4
\end{verbatim}

\begin{Shaded}
\begin{Highlighting}[]
\NormalTok{x <-}\StringTok{ }\KeywordTok{c}\NormalTok{(}\DecValTok{0}\NormalTok{, }\DecValTok{-3}\NormalTok{, }\DecValTok{-1}\NormalTok{, }\OtherTok{NA}\NormalTok{,}\DecValTok{4}\NormalTok{ ,}\DecValTok{5}\NormalTok{, }\DecValTok{8}\NormalTok{)}
\NormalTok{x <-}\StringTok{ }\NormalTok{x }\OperatorTok{+}\StringTok{ }\DecValTok{1}
\NormalTok{x}
\end{Highlighting}
\end{Shaded}

\begin{verbatim}
## [1]  1 -2  0 NA  5  6  9
\end{verbatim}

\begin{Shaded}
\begin{Highlighting}[]
\NormalTok{z <-}\StringTok{ }\NormalTok{x[(}\OperatorTok{!}\KeywordTok{is.na}\NormalTok{(x)) }\OperatorTok{&}\StringTok{ }\NormalTok{x }\OperatorTok{>}\StringTok{ }\DecValTok{1}\NormalTok{]}
\NormalTok{z}
\end{Highlighting}
\end{Shaded}

\begin{verbatim}
## [1] 5 6 9
\end{verbatim}

\hypertarget{ux6b63ux6574ux6578ux5411ux91cf}{%
\subparagraph{正整數向量}\label{ux6b63ux6574ux6578ux5411ux91cf}}

\begin{Shaded}
\begin{Highlighting}[]
\NormalTok{x <-}\StringTok{ }\DecValTok{10}\OperatorTok{:}\DecValTok{20}
\NormalTok{x}
\end{Highlighting}
\end{Shaded}

\begin{verbatim}
##  [1] 10 11 12 13 14 15 16 17 18 19 20
\end{verbatim}

\begin{Shaded}
\begin{Highlighting}[]
\NormalTok{y <-}\StringTok{ }\NormalTok{x[}\DecValTok{1}\OperatorTok{:}\DecValTok{5}\NormalTok{]}
\NormalTok{y}
\end{Highlighting}
\end{Shaded}

\begin{verbatim}
## [1] 10 11 12 13 14
\end{verbatim}

\begin{Shaded}
\begin{Highlighting}[]
\NormalTok{z <-}\StringTok{ }\NormalTok{x[ }\KeywordTok{c}\NormalTok{(}\DecValTok{1}\NormalTok{, }\DecValTok{2}\NormalTok{, }\DecValTok{3}\NormalTok{, }\DecValTok{8}\NormalTok{, }\DecValTok{9}\NormalTok{, }\DecValTok{9}\NormalTok{) ]}
\NormalTok{z}
\end{Highlighting}
\end{Shaded}

\begin{verbatim}
## [1] 10 11 12 17 18 18
\end{verbatim}

\hypertarget{ux8ca0ux6574ux6578ux5411ux91cf}{%
\subparagraph{負整數向量}\label{ux8ca0ux6574ux6578ux5411ux91cf}}

說明: 負數的意思是將指定的元素排除,將剩餘的元素選出

\begin{Shaded}
\begin{Highlighting}[]
\NormalTok{x <-}\StringTok{ }\DecValTok{10}\OperatorTok{:}\DecValTok{20}
\NormalTok{x}
\end{Highlighting}
\end{Shaded}

\begin{verbatim}
##  [1] 10 11 12 13 14 15 16 17 18 19 20
\end{verbatim}

\begin{Shaded}
\begin{Highlighting}[]
\NormalTok{y <-}\StringTok{ }\NormalTok{x[}\OperatorTok{-}\NormalTok{(}\DecValTok{1}\OperatorTok{:}\DecValTok{5}\NormalTok{)]}
\NormalTok{y}
\end{Highlighting}
\end{Shaded}

\begin{verbatim}
## [1] 15 16 17 18 19 20
\end{verbatim}

\hypertarget{ux5b57ux5143ux5411ux91cf}{%
\subparagraph{字元向量}\label{ux5b57ux5143ux5411ux91cf}}

說明 : 只能用在可以用 names 屬性區別其元素的向量,在使用之前必須以 names
函數先設定 names 屬性

\begin{Shaded}
\begin{Highlighting}[]
\NormalTok{fruit <-}\StringTok{ }\KeywordTok{c}\NormalTok{(}\DecValTok{5}\NormalTok{, }\DecValTok{10}\NormalTok{, }\DecValTok{1}\NormalTok{, }\DecValTok{20}\NormalTok{)}
\NormalTok{fruit}
\end{Highlighting}
\end{Shaded}

\begin{verbatim}
## [1]  5 10  1 20
\end{verbatim}

\begin{Shaded}
\begin{Highlighting}[]
\KeywordTok{names}\NormalTok{(fruit) <-}\StringTok{ }\KeywordTok{c}\NormalTok{(}\StringTok{"orange"}\NormalTok{, }\StringTok{"banana"}\NormalTok{, }\StringTok{"apple"}\NormalTok{, }\StringTok{"peach"}\NormalTok{)}
\NormalTok{fruit}
\end{Highlighting}
\end{Shaded}

\begin{verbatim}
## orange banana  apple  peach 
##      5     10      1     20
\end{verbatim}

\begin{Shaded}
\begin{Highlighting}[]
\NormalTok{lunch <-}\StringTok{ }\NormalTok{fruit[}\StringTok{"apple"}\NormalTok{]}
\NormalTok{lunch}
\end{Highlighting}
\end{Shaded}

\begin{verbatim}
## apple 
##     1
\end{verbatim}

\begin{Shaded}
\begin{Highlighting}[]
\NormalTok{dinner <-}\StringTok{ }\NormalTok{fruit[}\KeywordTok{c}\NormalTok{(}\StringTok{"orange"}\NormalTok{, }\StringTok{"peach"}\NormalTok{)]}
\NormalTok{dinner}
\end{Highlighting}
\end{Shaded}

\begin{verbatim}
## orange  peach 
##      5     20
\end{verbatim}

\hypertarget{ux4f7fux7528ux7d22ux5f15ux66f4ux65b0ux5143ux7d20ux503c}{%
\subparagraph{使用索引更新元素值}\label{ux4f7fux7528ux7d22ux5f15ux66f4ux65b0ux5143ux7d20ux503c}}

\begin{Shaded}
\begin{Highlighting}[]
\NormalTok{x <-}\StringTok{ }\KeywordTok{c}\NormalTok{(}\DecValTok{1}\NormalTok{, }\DecValTok{2}\NormalTok{, }\OtherTok{NA}\NormalTok{, }\OtherTok{NA}\NormalTok{, }\DecValTok{3}\NormalTok{)}
\NormalTok{x}
\end{Highlighting}
\end{Shaded}

\begin{verbatim}
## [1]  1  2 NA NA  3
\end{verbatim}

\begin{Shaded}
\begin{Highlighting}[]
\NormalTok{x[}\KeywordTok{is.na}\NormalTok{(x)] <-}\StringTok{ }\DecValTok{0}
\NormalTok{x}
\end{Highlighting}
\end{Shaded}

\begin{verbatim}
## [1] 1 2 0 0 3
\end{verbatim}

\hypertarget{whichux51fdux6578}{%
\subparagraph{\texorpdfstring{
which函數}{ which函數}}\label{whichux51fdux6578}}

\begin{Shaded}
\begin{Highlighting}[]
\NormalTok{x <-}\StringTok{ }\DecValTok{-5}\OperatorTok{:}\DecValTok{5}
\NormalTok{x}
\end{Highlighting}
\end{Shaded}

\begin{verbatim}
##  [1] -5 -4 -3 -2 -1  0  1  2  3  4  5
\end{verbatim}

\begin{Shaded}
\begin{Highlighting}[]
\KeywordTok{which}\NormalTok{(x }\OperatorTok\StringTok{ }\DecValTok{3} \OperatorTok{==}\StringTok{ }\DecValTok{2}\NormalTok{)}
\end{Highlighting}
\end{Shaded}

\begin{verbatim}
## [1]  2  5  8 11
\end{verbatim}

\begin{Shaded}
\begin{Highlighting}[]
\KeywordTok{which.min}\NormalTok{(x)}
\end{Highlighting}
\end{Shaded}

\begin{verbatim}
## [1] 1
\end{verbatim}

\begin{Shaded}
\begin{Highlighting}[]
\KeywordTok{which.max}\NormalTok{(x)}
\end{Highlighting}
\end{Shaded}

\begin{verbatim}
## [1] 11
\end{verbatim}

\hypertarget{ux91cdux8907ux5411ux91cf}{%
\subparagraph{重複向量}\label{ux91cdux8907ux5411ux91cf}}

\begin{Shaded}
\begin{Highlighting}[]
\DecValTok{1}\OperatorTok{:}\DecValTok{4}
\end{Highlighting}
\end{Shaded}

\begin{verbatim}
## [1] 1 2 3 4
\end{verbatim}

\begin{Shaded}
\begin{Highlighting}[]
\DecValTok{1}\OperatorTok{:}\DecValTok{8}
\end{Highlighting}
\end{Shaded}

\begin{verbatim}
## [1] 1 2 3 4 5 6 7 8
\end{verbatim}

\begin{Shaded}
\begin{Highlighting}[]
\DecValTok{1}\OperatorTok{:}\DecValTok{4} \OperatorTok{+}\StringTok{ }\DecValTok{1}\OperatorTok{:}\DecValTok{8}
\end{Highlighting}
\end{Shaded}

\begin{verbatim}
## [1]  2  4  6  8  6  8 10 12
\end{verbatim}

\begin{Shaded}
\begin{Highlighting}[]
\DecValTok{1}\OperatorTok{:}\DecValTok{2} \OperatorTok{+}\StringTok{ }\DecValTok{1}\OperatorTok{:}\DecValTok{7}
\end{Highlighting}
\end{Shaded}

\begin{verbatim}
## Warning in 1:2 + 1:7: 較長的物件長度並非較短物件長度的倍數
\end{verbatim}

\begin{verbatim}
## [1] 2 4 4 6 6 8 8
\end{verbatim}

\hypertarget{ux4f7fux7528repux51fdux6578ux5efaux7acbux91cdux8986ux5411ux91cf}{%
\subparagraph{使用rep函數建立重覆向量}\label{ux4f7fux7528repux51fdux6578ux5efaux7acbux91cdux8986ux5411ux91cf}}

\begin{Shaded}
\begin{Highlighting}[]
\KeywordTok{rep}\NormalTok{(}\DecValTok{1}\OperatorTok{:}\DecValTok{3}\NormalTok{, }\DecValTok{5}\NormalTok{)  }\CommentTok{#rep(要重複向量,重複次數)}
\end{Highlighting}
\end{Shaded}

\begin{verbatim}
##  [1] 1 2 3 1 2 3 1 2 3 1 2 3 1 2 3
\end{verbatim}

\hypertarget{ux9663ux5217array}{%
\subsubsection{\texorpdfstring{ 02.
陣列Array}{ 02. 陣列Array}}\label{ux9663ux5217array}}

\hypertarget{ux56e0ux5b50factor}{%
\subsubsection{\texorpdfstring{ 03.
因子Factor}{ 03. 因子Factor}}\label{ux56e0ux5b50factor}}

\hypertarget{ux5217ux8868list}{%
\subsubsection{\texorpdfstring{
04.列表LIST}{ 04.列表LIST}}\label{ux5217ux8868list}}

\hypertarget{ux4f7fux7528data.frameux5efaux7acblist}{%
\paragraph{使用data.frame()建立List}\label{ux4f7fux7528data.frameux5efaux7acblist}}

\begin{Shaded}
\begin{Highlighting}[]
\NormalTok{StuDF <-}\StringTok{ }\KeywordTok{data.frame}\NormalTok{(}\DataTypeTok{SID=}\KeywordTok{c}\NormalTok{(}\DecValTok{1}\NormalTok{,}\DecValTok{2}\NormalTok{,}\DecValTok{3}\NormalTok{,}\DecValTok{4}\NormalTok{,}\DecValTok{5}\NormalTok{),}\DataTypeTok{NAME=}\KeywordTok{c}\NormalTok{(}\StringTok{"魯夫"}\NormalTok{,}\StringTok{"索隆"}\NormalTok{,}\StringTok{"香吉士"}\NormalTok{,}\StringTok{"騙人布"}\NormalTok{,}\StringTok{"娜美"}\NormalTok{),}\DataTypeTok{SCORE=}\KeywordTok{c}\NormalTok{(}\DecValTok{50}\NormalTok{,}\DecValTok{60}\NormalTok{,}\DecValTok{80}\NormalTok{,}\DecValTok{80}\NormalTok{,}\DecValTok{90}\NormalTok{))}
\NormalTok{StuDF }
\end{Highlighting}
\end{Shaded}

\begin{verbatim}
##   SID   NAME SCORE
## 1   1   魯夫    50
## 2   2   索隆    60
## 3   3 香吉士    80
## 4   4 騙人布    80
## 5   5   娜美    90
\end{verbatim}

\hypertarget{ux986fux793aux6b04ux4f4dux540dux7a31}{%
\paragraph{顯示欄位名稱}\label{ux986fux793aux6b04ux4f4dux540dux7a31}}

\begin{Shaded}
\begin{Highlighting}[]
\KeywordTok{colnames}\NormalTok{(StuDF)}
\end{Highlighting}
\end{Shaded}

\begin{verbatim}
## [1] "SID"   "NAME"  "SCORE"
\end{verbatim}

\hypertarget{ux4f7fux7528strux51fdux6578ux986fux793aux6b04ux4f4dux4e4bux8cc7ux6599ux578bux5225}{%
\paragraph{使用str()函數,顯示欄位之資料型別}\label{ux4f7fux7528strux51fdux6578ux986fux793aux6b04ux4f4dux4e4bux8cc7ux6599ux578bux5225}}

\begin{Shaded}
\begin{Highlighting}[]
\KeywordTok{str}\NormalTok{(StuDF) }
\end{Highlighting}
\end{Shaded}

\begin{verbatim}
## 'data.frame':    5 obs. of  3 variables:
##  $ SID  : num  1 2 3 4 5
##  $ NAME : Factor w/ 5 levels "香吉士","娜美",..: 4 3 1 5 2
##  $ SCORE: num  50 60 80 80 90
\end{verbatim}

\hypertarget{ux4f7fux7528ux7b26ux865fux505aux6b04ux4f4dux8cc7ux6599ux64f7ux53d6}{%
\paragraph{使用\$符號做欄位資料擷取}\label{ux4f7fux7528ux7b26ux865fux505aux6b04ux4f4dux8cc7ux6599ux64f7ux53d6}}

\begin{Shaded}
\begin{Highlighting}[]
\NormalTok{StuDF}\OperatorTok{$}\NormalTok{SID}
\end{Highlighting}
\end{Shaded}

\begin{verbatim}
## [1] 1 2 3 4 5
\end{verbatim}

\begin{Shaded}
\begin{Highlighting}[]
\KeywordTok{head}\NormalTok{(StuDF}\OperatorTok{$}\NormalTok{SID)}
\end{Highlighting}
\end{Shaded}

\begin{verbatim}
## [1] 1 2 3 4 5
\end{verbatim}

\end{document}
